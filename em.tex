\documentclass{article}
\usepackage{graphicx} % Required for inserting images
\usepackage{yehyun}
\usepackage{xcolor}
\usepackage{physics}

\title{Notes on Electrodynamics}
\author{Yehyun Choi}

\begin{document}

\maketitle

\pagebreak

\section{Preliminaries}
\textcolor{red}{Goldbarch Ch.2}
\paragraph{Complete orthonormal functions}
A set of functions $\{U_n\}\in L^2(S)$ (the inner product is $\langle f,g\rangle=\int_Swf^*g\dd x$, where $w$ is the weighing function) is \textbf{orthonormal} if 
$$\langle U_n,U_m\rangle=\delta_{nm}.$$
The set $\{U_n\}\in L^2(S)$ is \textbf{complete} if minimum square error $E_n=\norm{f-(c_1U_1+\cdots c_nU_n)}_S$ approaches zero as $n\to\infty$. Therefore, we have a convergent expansion and completeness relation
$$f(\xi)=\sum^\infty_{n=1}a_nU_n(\xi),\quad a_n=\int_SU^*_n(\xi)f(\xi)\dd\xi\implies \sum^\infty_{n=1}U^*_n(\xi')U_n(\xi)=\delta(\xi'-\xi),$$
For example, $U_k=(2\pi)^{-1}\exp(ikx)$ is complete and orthonormal in $L^2(\mathbb R)$:
$$f(x)=\frac 1{\sqrt{2\pi}}\int^\infty_{-\infty}A(k)e^{ikx}\dd k,\quad A(k)=\frac 1{\sqrt{2\pi}}\int^\infty_{-\infty}e^{-ikx}f(x)\dd x.$$
Orthogonality and completeness relations are 
$$\frac 1{2\pi}\int^\infty_{-\infty}e^{i(k-k')x}\dd x=\delta(k-k'),\quad \frac 1{2\pi}\int^\infty_{-\infty}e^{ik(x-x')}\dd k=\delta(x-x').$$



\paragraph{Green's functions}
Green's function, for Poisson's equation, is defined 
$$\nabla'^2G(\mathbf x,\mathbf x')=-4\pi\delta(\mathbf x-\mathbf x')\implies G(\mathbf x,\mathbf x')=\frac 1{\abs{\mathbf x-\mathbf x'}}+F(\mathbf x,\mathbf x'),\hspace{10mm}\nabla'^2F(\mathbf x,\mathbf x)=0.$$
From Green's theorem, 
$$\Phi(\mathbf x)=\frac 1{4\pi\epsilon_0}\int_V\rho(\mathbf x')G(\mathbf x,\mathbf x')\dd^3x'+\frac 1{4\pi}\oint_S\left[G(\mathbf x,\mathbf x')\pde\Phi{n'}-\Phi(\mathbf x')\pde{G(\mathbf x,\mathbf x')}{n'}\right]\dd a'.$$
From the free parameter $F(\mathbf x,\mathbf x')$, we can let either $G(\mathbf x,\mathbf x')=0$ or $\pde{G(\mathbf x,\mathbf x')}{n'}=0.$ 

\paragraph{Uniqueness theorem}
Let $U=\Phi_2-\Phi_1$ where $\Phi_1$ and $\Phi_2$ both satisfy the same boundary conditions. From Green's first theorem, 
$$\int_V(U\nabla^2U+\nabla U\cdot\nabla U)\dd^3x=\oint_SU\pde Un\,\dd a.$$
For Dirichlet ($U=0$ on S) or Neumann ($\pde Un=0$ on S)conditions,
$$\int_V\abs{\nabla U}^2\dd^3x=0,$$
meaning $U$ is a constant. 


\pagebreak

\section{Elecrostatics}
\textcolor{red}{Jackson Chapter 1 (except 1.12-1.13)}
\subsection{Axioms and Definitions}

\paragraph{Coulomb's law}

Coulomb's law states: the force between two small charged bodies separated in air a distance large compared to their dimensions
\begin{enumerate}
    \item varies directly as the magnitude of each charge,
    \item varies inversely as the square of the distance between them,
    \item is directed along the line joining the bodies, and
    \item is attractive if the bodies are oppositely charged and repulsive if the bodies have the same type of charge.
\end{enumerate}
This can be written as 
$$\mathbf E=kq_1\frac{\mathbf x_1-\mathbf x_2}{\abs{\mathbf x_1-\mathbf x_2}^3}.$$
where $\mathbf F=q\mathbf E$ by definition. 

Electrical forces follow the principle of superposition; for a discrete/continuous distribution of charges,
$$\mathbf E=\frac 1{4\pi\epsilon_0}\sum^n_{i=1}q_i\frac{\mathbf x-\mathbf x_i}{\abs{\mathbf x-\mathbf x_i}^3}=\frac 1{4\pi\epsilon_0}\int\rho(\mathbf x')\frac{\mathbf x-\mathbf x'}{\abs{\mathbf x-\mathbf x'}^3}\dd^3x'.$$

\paragraph{Gauss' law}
From Coulomb's law,
$$\mathbf E\cdot\mathbf n\,\dd a=\frac q{4\pi\epsilon_0}\frac{\cos\theta}{r^2}\dd a=\frac q{4\pi\epsilon_0}\dd\Omega,$$
for some surface $S$. Therefore,
$$\oint_S\mathbf E\cdot\,\dd a=\frac q{\epsilon_0}\implies\nabla\cdot\mathbf E=\rho/\epsilon_0,$$
from the divergence theorem.
\paragraph{Potential}
We can write the Coulomb's law as 
$$\mathbf E=-\nabla\left(\frac 1{4\pi\epsilon_0}\int\frac{\rho(\mathbf x')}{\abs{\mathbf x-\mathbf x'}}\dd^3x'\right)=-\nabla\Phi,$$
where $\Phi$ is some scalar quantity. We define this to be the electric potential.


\paragraph{Boundary conditions}

For a charged sheet, the discontinuity in the electrical field is 
$$(\mathbf E_2-\mathbf E_1)\cdot\mathbf n=\sigma/\epsilon_0.$$
For a charged dipole layer \textcolor{red}{Jackson page 32}, we have
$$\Phi_2-\Phi_1=D/\epsilon_0.$$

\paragraph{Poisson equations}

Poisson's equation states that
$$\nabla^2\Phi=-\rho/\epsilon_0.$$
It is easy to see that $\Phi(\mathbf x)=\frac 1{4\pi\epsilon_0}\int\frac{\rho(\mathbf x')}{\abs{\mathbf x-\mathbf x'}}\dd^3x'$ satisfies Poisson's equation using $\div(1/r)=4\pi\delta(\mathbf x).$


\paragraph{Energy} Consider a charge brought from infinity. The work done is 
$$W=-\int\mathbf F\cdot\dd\mathbf l=q\int\nabla\Phi\cdot\dd\mathbf l=q\Phi(\mathbf x),$$
where we took the potential at infinity to zero. It can be written that, for a charge distribution,
\begin{align*}
    W&=\frac 1{8\pi\epsilon_0}\sum_i\sum_j\frac{q_iq_j}{\abs{\mathbf x_i-\mathbf x_j}}\\
    &=\frac 1{8\pi\epsilon_0}\iint \frac{\rho(\mathbf x)\rho(\mathbf x')}{\abs{\mathbf x-\mathbf x'}}\,\dd^3x\,\dd^3x'\\
    &=\frac 12\int\rho(\mathbf x)\Phi(\mathbf x)\,\dd^3x\\
    &=\frac{-\epsilon_0}2\int\Phi\nabla^2\Phi\,\dd^3x\\
    &=\frac{\epsilon_2}2\int_{\mathbb R^3}\abs{\nabla\Phi}^2\,\dd^3x,
\end{align*}
where we have used Green's first theorem for the last part. From the last result, the energy density is $w=\frac{\epsilon_0}2\abs{\mathbf E}^2.$ However, be warned that this also contains self energy contributions - the interaction potential energy, however, is meaningful in this case \textcolor{red}{(Jackson p. 42.)}

\paragraph{Green's theorems}
The first Green's theorem can be obtained by applying the div. theorem on $\mathbf A=\phi\nabla\psi$, 
$$\int_V\nabla\cdot(\phi\nabla\psi)\,\dd^3x=\oint_S\phi\nabla\psi\cdot\mathbf n\,\dd a\implies\int_V\phi\nabla^2\psi+\nabla\phi\cdot\nabla\psi\,\dd^3x=\oint_S\phi\pde\psi n\dd a.$$
Taking the antisymmetric part, we obtain the second Green's theorem:
$$\int_V(\phi\nabla^2\psi-\psi\nabla^2\phi)\dd^3x=\oint_S\left[\phi\ode\psi n-\psi\ode\phi n\right]\dd a.$$
As an example, $\phi=\Phi$ and $\psi=1/\abs{\mathbf x-\mathbf x'}$. We get
$$\int_V\left[-4\pi\Phi(\mathbf x')\delta(\mathbf x-\mathbf x')+\frac 1{\epsilon_0R}\rho(\mathbf x')\right]\dd^3x'=\oint_S\left[\Phi\pde{}{n'}\left(\frac 1R\right)-\frac 1R\pde\Phi{n'}\right]\dd a'.$$
Therefore,
$$\Phi(\mathbf x)=\frac 1{4\pi\epsilon_0}\int_V\frac{\rho(\mathbf x')}{R}\dd^3x'+\frac 1{4\pi}\oint_S\left[\frac 1R\pde\Phi{n'}-\Phi\pde{}{n'}\left(\frac 1R\right)\right]\dd a'.$$


\subsection{Method of Images}
\textcolor{red}{Jackson Chapter 2-3}

\paragraph{Conducting Sphere - Point Charge}
Consider a grounded conducting sphere of radius $a$ centered at origin and a point charge $q$ at $\mathbf y$. If we let the image charge be at $\mathbf y'$ with charge $q'$, the potential at $\mathbf x$ is 
$$\Phi(\mathbf x)=\frac{q/4\pi\epsilon_0}{a\abs{\mathbf n-\frac ya\mathbf n'}}+\frac{q'/4\pi\epsilon_0}{y'\abs{\mathbf n'-\frac a{y'}\mathbf n}},$$
where we let $\mathbf y$ and $\mathbf y'$ lie on $\mathbf n'$, a unit vector and $\mathbf x$ on $\mathbf n$. This has the solution
$$q'=-\frac ayq,\quad y'=\frac{a^2}y.$$
The force acting on charge $q$ can be calculated via Coulomb's law or by integrating $(\sigma^2/2\epsilon_0)\dd a$:
$$\abs{\mathbf F}=\frac 1{4\pi\epsilon_0}\frac{q^2}{a^2}\left(\frac ay\right)^3\left(1-\frac{a^2}{y^2}\right)^{-2}.$$

If the total charge on the insulated sphere be $Q$. We then use linear superposition of induced charge, $q'$ with the ``added'' charge $Q-q'$ as evenly spread over the surface. Hence,
$$\Phi(\mathbf x)=\frac 1{4\pi\epsilon_0}\left[\frac q{\abs{\mathbf x-\mathbf y}}-\frac{aq}{y\abs{\mathbf x-\frac{a^2}{y^2}\mathbf y}}+\frac{Q+\frac ayq}{\abs{\mathbf x}}\right]$$

\paragraph{Conducting Sphere - Uniform Field}
One can think of uniform field as two opposite charges very far away. We get
$$\Phi=-E_0\left(r-\frac{a^3}{r^2}\right)\cos\theta\implies\sigma=-\epsilon_0\left.\pde\Phi r\right|_{r=a}=3\epsilon_0E_0\cos\theta.$$

\paragraph{General Solution}
The green's function (for dirichlet boundary condition) is evidently 
$$G(\mathbf x,\mathbf x')=\frac 1{\abs{\mathbf x-\mathbf x'}}-\frac a{x'\abs{\mathbf x-\frac{a^2}{x'^2}\mathbf x'}}=\frac 1{(x^2+x'^2-2xx'\cos\gamma)^{1/2}}-\frac 1{\left(\frac{x^2x'^2}{a^2}+a^2-2xx'\cos\gamma\right)}.$$
Taking the derivative and applying Green's theorem, we obtain
$$\Phi(\mathbf x)=\frac 1{4\pi}\int\Phi(a,\theta',\phi')\frac{a(x^2-a^2)}{(x^2+a^2-2ax\cos\gamma)^{3/2}}\dd\Omega',$$
where $\gamma$ is the angle between $x$ and $x'$.
Using this, one can find the potential of a conducting sphere with hemispheres at different potentials (\textcolor{red}{Jackson 2.7}).


\subsection{Separation of Variables}
If we let the solution of a PDE be a product of single variabled functions, we can reduce the PDE to a system of ODEs. This is particularly useful in solving Laplace's equation.

\paragraph{Cartesian}
We have 
$$\Phi=XYZ\implies \frac 1X\oden 2Xx+\frac 1Y\oden 2Yy+\frac 1Z\oden 2Zz=0.$$
We get 
$$\frac 1X\oden 2Xx=-\alpha^2,\quad \frac 1Y\oden 2Yy=-\beta^2,\quad \frac 1Z\oden 2Zz=\gamma^2,$$
where $\alpha^2+\beta^2=\gamma^2.$ The general solution is $\Phi=e^{\pm i\alpha x}e^{\pm i\beta y}e^{\pm\sqrt{\alpha^2+\beta^2}z}.$ 
To find the specific solution, one must impose the boundary conditions. It is also useful to superpose such solutions, which orthonormality of $e^{ix}$ comes in handy. (\textcolor{red}{Jackson 2.9, 2.10, 2.11})

\paragraph{Spherical}
With $\Phi=\frac UrPQ,$ (reminder that you have to divide by $r$!) we get the differential equations
\begin{align*}
    \frac 1Q\oden 2Q\phi&=-m^2\implies Q=e^{im\phi}\\
    \frac 1{\sin\theta}\ode{}\theta\left(\sin\theta\ode P\theta\right)+\left[l(l+1)-\frac{m^2}{\sin^2\theta}\right]P&=0\implies P = P^m_l(\cos\theta)\\
    \oden 2Ur-\frac{l(l+1)}{r^2}U&=0\implies U=Ar^{l+1}+Br^{-l},
\end{align*}
where $m$, $l$ are constants.
\paragraph{Spherical - Azimuthal Symmetry}
For $m=0$, the solution for $P$ are the legendre polynomials. The first few polynomials are 
\begin{align*}
    P_0&=1\\
    P_1&=x\\
    P_2&=\frac 12(3x^2-1)\\
    P_3&=\frac 12(5x^3-3x)\\
    P_4&=\frac 18(35x^4-30x^2+3)\\
    P_5&=\frac 18(63x^5-70x^3+15x).
\end{align*}

Rodrigues' formula in particularly useful:
$$P_l(x)=\frac 1{2^ll!}\oden l{}x(x^2-1)^l.$$
Furthermore, the legendre polynomials are orthonormal in $(1,-1)$ if you let $U_l(x)=\sqrt{\frac{2l+1}2}P_l(x).$ Therefore, one can expand a function:
$$f(x)=\sum^\infty_{l=0}A_lP_l(x),\quad A_l=\frac{2l+1}2\int^1_{-1}f(x)P_l(x)\dd x.$$
In other words, given the general solution 
$$\Phi(r,\theta)=\sum^\infty_{l=0}[A_lr^l+B_lr^{-(l+1)}]P_l(\cos\theta),$$
one cam impose boundary condition since, on a sphere, for example, the boundary condition becomes $V(\theta)=\sum^\infty_{l=0}A_la^lP_l(\cos\theta)\implies A_l=\frac{2l+1}{2a^l}\int^\pi_0V(\theta)P_l(\cos\theta)\sin\theta\dd\theta.$

If we have an azimuthally symmetric system, due to the uniqueness theorem, the potential on the symmetric axis gives the general potential, when multiplied by legendre polynomials; given $\Phi(\theta=0)=\sum^\infty_{l=0}[A_lr^l+B_lr^{-(l+1)}]\implies\Phi(r,\theta)=\sum^\infty_{l=0}[A_lr^l+B_lr^{-(l+1)}]P_l(\cos\theta)$. 

A useful application of this is the expansion of the $1/\abs{\mathbf x-\mathbf x'}$:
$$\frac 1{\abs{\mathbf x-\mathbf x'}}=\sum^\infty_{l=0}\frac{r^l_<}{r^{l+1}_>}P_l(\cos\gamma),$$
where $\gamma$ is the angle between $\mathbf x$ and $\mathbf x'$ and $r_>$ and $r_<$ are the greater and smaller of $\mathbf x$ and $\mathbf x'$, respectively. 
\paragraph{Spherical - General}
For $m\in[-l,l]$, $P$ has the formula
$$P^m_l(x)=\frac{(-1)^m}{2^ll!}(1-x^2)^{m/2}\oden{l+m}{}x(x^2-1)^l.$$
The associated legendre polynomials are again orthogonal in $\phi\in[0,2\pi]$ and $\cos\theta\in[-1,1]$. The normalized associated Legendre polynomials, called the spherical harmonics, are defined by 
$$Y_{lm}(\theta,\phi)=\sqrt{\frac{2l+1}{4\pi}\frac{(l-m)!}{(l+m)!}}P^m_l(\cos\theta)e^{im\phi},\quad Y_{l0}(\theta,\phi)=\sqrt{\frac{2l+1}{4\pi}}P_l(\cos\theta).$$
The first few terms can be written out:
\begin{align*}
    Y_{00}&=\sqrt{\frac 1{4\pi}}\\
    Y_{1,\pm 1}&=\mp\sqrt{\frac 3{8\pi}}\sin\theta e^{\pm i\phi},\quad Y_{10}=\sqrt{\frac 3{4\pi}}\cos\theta\\
    Y_{2,\pm 2}&=\frac 14\sqrt{\frac{15}{2\pi}}\sin^2\theta e^{\pm 2i\phi},\quad Y_{2,\pm 1}=\mp\frac 12\sqrt{\frac{15}{2\pi}}\sin\theta\cos\theta e^{\pm i\phi},\quad Y_{2,0}=\frac 14\sqrt{\frac 5\pi}(3\cos^2\theta-1)\\
    Y_{3,\pm 3}&=\mp\frac 18\sqrt{\frac{35}\pi}\sin^3\theta e^{\pm 3i\phi},\quad Y_{3,\pm 2}=\frac 14\sqrt{\frac{105}{2\pi}}\sin^2\theta\cos\theta e^{\pm 2i\phi},\quad Y_{3,\pm 1}=\mp\frac 18\sqrt{\frac{21}\pi}\sin\theta(5\cos^2\theta-1)e^{\pm i\phi}\\
    Y_{3,0}&=\frac 14\sqrt{\frac 7\pi}(5\cos^3\theta-3\cos\theta).
\end{align*}

Its orthonormality and completeness relations state 
\begin{align*}
    &\int^{2\pi}_0\dd\phi\int^\pi_0\sin\theta\dd\theta Y^*_{l'm'}(\theta,\phi)Y_{lm}(\theta,\phi)=\delta_{l'l}\delta_{m'm}\\
    &\sum^\infty_{l=0}\sum^l_{m=-l}Y^*_{lm}(\theta',\phi')Y_{lm}(\theta,\phi)=\delta(\phi-\phi')\delta(\cos\theta-\cos\theta').
\end{align*}
Evidently, an arbitrary function can be expanded into spherical harmonics:
$$g(\theta,\phi)=\sum^\infty_{l=0}\sum^l_{m=-l}A_{lm}Y_{lm}(\theta,\phi),\quad A_{lm}=\int\dd\Omega Y^*_{lm}(\theta,\phi)g(\theta,\phi).$$
Lastly, the addition theorem for spherical harmonics expresses a Legendre polynomial of order $l$ in the angle $\gamma$ in products of the spherical harmonics of the angles $\theta,\phi$ and $\theta',\phi'$:
$$P_l(\cos\gamma)=\frac{4\pi}{2l+1}\sum^l_{m=-l}Y^*_{lm}(\theta',\phi')Y_{lm}(\theta,\phi).$$
Therefore, $1/\abs{\mathbf x-\mathbf x'}$ can be written
$$\frac 1{\abs{\mathbf x-\mathbf x'}}=4\pi\sum^\infty_{l=0}\sum^l_{m=-l}\frac 1{2l+1}\frac{r^l_{<}}{r^{l+1}_>}Y^*_{lm}(\theta',\phi')Y_{lm}(\theta,\phi).$$

\paragraph{Cylindrical} With $\Phi=RQZ$, we get the differential equations
\begin{align*}
    \oden 2Zz-k^2Z&=0\implies Z=e^{\pm kz}\\
    \oden 2Q\phi+\nu^2Q&=0\implies Q=e^{\pm i\nu\phi}\\
    \oden 2R\rho+\frac 1\rho\ode R\rho+\left(k^2-\frac{\nu^2}{\rho^2}\right)R&=0\implies R=AJ_\nu(k\rho)+BN_m(k\rho).
\end{align*}

The radial equation, after a change of variable $x=k\rho$, becomes the Bessel equation ($\oden 2Rx+\frac 1x\ode Rx+\left(1-\frac{\nu^2}{x^2}\right)R=0$). The two solutions are (Bessel functions of the first kind of order $\pm\nu$)
$$J_\nu(x)=\left(\frac x2\right)^\nu\sum^\infty_{j=0}\frac{(-1)^j}{j!\Gamma(j+\nu+1)}\left(\frac x2\right)^{2j},\quad J_{-\nu}(x)=\left(\frac x2\right)^{-\nu}\sum^\infty_{j=0}\frac{(-1)^j}{j!\Gamma(j-\nu+1)}\left(\frac x2\right)^{2j}.$$
The Bessel function can be written as an integral:
$$J_n(z)=\frac{i^{-n}}{\pi}\int^\pi_0e^{iz\cos\theta}(n\theta)\dd\theta.$$


For $\nu\ne\mathbb Z$, these two solutions are linearly dependent. It is then customary to use $J_\nu(x)$ and $N_\nu(x)$ - Bessel function of the second kind:
$$N_\nu(x)=\frac{J_\nu(x)\cos\nu\pi-J_{-\nu}(x)}{\sin\nu\pi}.$$
For boundary condition purposes, for $x\ll 1$, $J_\nu(x)\approx x^\nu,\quad N_\nu(x)\approx \ln(x/2)$, for $\nu=0$, and $N_\nu(x)\approx x^{-\nu}$ for $\nu\ne 0$. For $x\gg 1$, $J_\nu(x)\approx\cos x,\quad N_\nu(x)\approx\sin x$.

A third set of solutions exist, called the Hankel functions, but are most commonly used in propagation of waves.

Lastly, only Bessel functions of the first kind are orthogonal on the interval $[0,1]$: the Fourier-Bessel series states 
$$f(\rho)=\sum^\infty_{n=1}A_{\nu n}J_\nu\left(x_{\nu n}\frac\rho a\right),\quad A_{\nu n}=\frac 2{a^2J^2_{\nu+1}(x_{\nu n})}\int^a_0\rho f(\rho)J_\nu\left(\frac{x_{\nu n}\rho}a\right)\dd\rho.$$

\subsection{Green's Function}
Green's function can be found by expanding both sides of $\nabla^2G(\mathbf x,\mathbf x')=-4\pi\delta(\mathbf x-\mathbf x')$ - left side can be normally expanded and the right side can be expanded using completeness of spherical harmonics/Bessel functions. 
\paragraph{Spherical Green Function}
The Green function for a spherical shell bounded by $r=a$ and $r=b$ is 
$$G(\mathbf x,\mathbf x')=4\pi\sum^\infty_{l=0}\sum^l_{m-l}\frac{Y^*_{lm}(\theta',\phi')Y_{lm}(\theta,\phi)}{(2l+1)\left[1-\left(\frac ab\right)^{2l+1}\right]}\left(r^l_<-\frac{a^{2l+1}}{r^{l+1}_<}\right)\left(\frac 1{r^{l+1}_>}-\frac{r^l_>}{b^{2l+1}}\right).$$
\paragraph{Eigenfunction Expansions}
Consider an elliptic differential equation of the form 
$$\nabla^2\psi_n(\mathbf x)+[f(\mathbf x)+\lambda_n]\psi_n(\mathbf x)=0.$$
Because $\nabla^2+f(\mathbf x)$ is a self-adjoint operator, its eigenfunctions are orthogonal:
$$\int_V\psi^*_m(\mathbf x)\psi_n(\mathbf x)\dd^3x=\delta_{mn}.$$
Hence, if we want to solve the Green function for the equation
$$\nabla^2G(\mathbf x,\mathbf x')+[f(\mathbf x)+\lambda]G(\mathbf x,\mathbf x')=-4\pi\delta(\mathbf x-\mathbf x'),$$
we can use $G(\mathbf x,\mathbf x')=\sum_na_n(\mathbf x')\psi_n(\mathbf x).$ If we multiply both sides by $\psi^*_m(\mathbf x'$ and integrate, we get $G(\mathbf x,\mathbf x')=4\pi\sum_n\frac{\psi^*_n(\mathbf x')\psi_n(\mathbf x)}{\lambda_n-\lambda}.$

\subsection{Multipole Expansion}
Comparing the separation of variables solution outside a sphere with the potential of a charge density distribution gives us the integral 
$$\Phi(\mathbf x)=\frac 1{\epsilon_0}\sum_{l,m}\frac 1{2l+1}q_{lm}\frac{Y_{lm}(\theta,\phi)}{r^{l+1}},\quad q_{lm}=\int Y^*_{lm}(\theta',\phi'){r'}^l\rho(\mathbf x')\dd^3x'$$
We call these coefficients $q_{lm}$ be called multipole moments. The first few terms are 
$$\Phi(\mathbf x)=\frac 1{4\pi\epsilon_0}\left[\frac qr+\frac{\mathbf p\cdot\mathbf x}{r^3}+\frac 12\sum_{i,j}Q_{ij}\frac{x_ix_j}{r^5}+\cdots\right],$$
where 
$$q=\rho(\mathbf x')\dd^3x',\quad \mathbf p=\int\mathbf x'\rho(\mathbf x')\dd^3x',\quad Q_{ij}=\int(3x'_ix'_j-{r'}^2\delta_{ij})\rho(\mathbf x')\dd^3x'.$$
Note,
\begin{enumerate}
    \item There's $(l+1)(l+2)/2$ cartesian multipole moments versus $(2l+1)$ spherical moments. This is becuase Cartesian multipole moments are traceless symmetric tensors.
    \item Multipole moments are dependent on the location on the origin.
\end{enumerate}

For the dipole term, in specific, 
$$\mathbf E(\mathbf x)=\frac 1{4\pi\epsilon_0}\left[\frac{3\mathbf n(\mathbf p\cdot\mathbf n)-\mathbf p}{\abs{\mathbf x-\mathbf x_0}^3}-\frac{4\pi}3\mathbf p\delta(\mathbf x-\mathbf x_0)\right].$$
Furthermore, the energy of a multipole system takes the form 
$$W=q\Phi(0)-\mathbf p\cdot\mathbf E(0)-\frac 16\sum_i\sum_jQ_{ij}\pde{E_j}{x_i}(0)+\cdots.$$

\subsection{Macroscopic Media}
We define the macroscopic (electric) polarization $\mathbf P$ and charge density $\rho$ as  
$$\mathbf P(\mathbf x)=\sum_iN_i\ev{\mathbf p_i},\quad \rho(\mathbf x)=\sum_iN_i\ev{e_i}+\rho_\text{excess},$$
where we're taking the weighted average of the dipole moment/molecular excess charge of the molecules in a small volume centered at $\mathbf x$ - $N_i$ are the weights of the molecules. The potential is 
$$\Phi(\mathbf x)=\frac 1{4\pi\epsilon_0}\int\dd^3x'\left[\frac{\rho(\mathbf x')}{\abs{\mathbf x-\mathbf x'}}+\mathbf P(\mathbf x')\cdot\nabla'\left(\frac 1{\abs{\mathbf x-\mathbf x'}}\right)\right]=\frac 1{4\pi\epsilon_0}\int\dd^3x'\frac 1{\abs{\mathbf x-\mathbf x'}}\left[\rho(\mathbf x')-\nabla'\cdot\mathbf P(\mathbf x')\right].$$
Defining the electric displacement $\mathbf D=\epsilon_0\mathbf E+\mathbf P$, we have $\nabla\cdot\mathbf D=\rho$. 

For a linearly dielectric medium, the polarization is parallel to $\mathbf E$:
$$\mathbf P=\epsilon_0\chi_e\mathbf E\implies \mathbf E=\epsilon\mathbf E,\quad \epsilon=\epsilon_0(1+\chi_e)\implies\nabla\cdot\mathbf E=\rho/\epsilon.$$
where $\chi_e$ is the susceptibility of the medium and $\epsilon/\epsilon_0=1+\chi_e$ is the dielectric constant.

\textbf{Boundary Problems}
The boundary conditions are 
$$(\mathbf D_2-\mathbf D_1)\cdot\mathbf n_{21}=\sigma,\quad(\mathbf E_2-\mathbf E_1)\times\mathbf n_{21}=0.$$
As an example, \textcolor{red}{Jackson 4.4} 

\section{Magnetostatics}
\subsection{Definitions}
For magnetostatics, we have $\nabla\cdot\mathbf J=0$. 

\paragraph{Biot-Savart Law}
The Biot-Savart law is given as 
$$\dd\mathbf F=I_1(\dd\mathbf I_1\times\mathbf B),\quad\dd\mathbf B=kI\frac{(\dd\mathbf I\times\mathbf x)}{\abs{\mathbf x}^3},$$
with $k=\mu_0/4\pi$ in SI units. From the force law, 
$$\mathbf F=\int\mathbf J(\mathbf x)\times\mathbf B(\mathbf x)\dd^3x,\quad\mathbf N=\int\mathbf x\times(\mathbf J\times\mathbf B)\dd^3x.$$
\paragraph{Ampere's law and potential(s)} For a current density, we have 
$$\mathbf B=\frac{\mu_0}{4\pi}\int\mathbf J(\mathbf x')\times\frac{\mathbf x-\mathbf x'}{\abs{\mathbf x-\mathbf x'}^3}\dd^3x'=\frac{\mu_0}{4\pi}\nabla\times\int\frac{\mathbf J(\mathbf x')}{\abs{\mathbf x-\mathbf x'}}\dd^3x'.$$
Evidently, $\nabla\cdot\mathbf B=0$. If we take the curl of the magnetic field, we obtain
$$\nabla\times\mathbf B=\mu_0\mathbf J\implies\oint_C\mathbf B\cdot\dd\mathbf l=\mu_0I.$$
From the fact that $\nabla\cdot\mathbf B=0$, we define the vector potential:
$$\mathbf B(\mathbf x)=\nabla\times\mathbf A(\mathbf x)\implies\mathbf A(\mathbf x)=\frac{\mu_0}{4\pi}\int\frac{\mathbf J(\mathbf x')}{\abs{\mathbf x -\mathbf x'}}\dd^3x'+\nabla\Psi(\mathbf x).$$
There's a degree of freedom in the vector potential and the related transformations are called gauge transformations - $\mathbf A\to\mathbf A+\nabla\Psi$. It is convenient to choose the gauge (Coulomb gauge) such that $\nabla\cdot\mathbf A=0\implies\nabla^2\mathbf A=-\mu_0\mathbf J$. $\Psi$ is constant for this gauge, because $\nabla\cdot\mathbf =0\implies\nabla^2\Psi=0$.

However, in the case that $\mathbf J=0$, we can define the magnetic scalar potential such that $\mathbf H=-\nabla\Phi_M$. If $\mu$ is piecewise constant, each region satisfies the Laplace equation: $\nabla^2\Phi_M=0$.

\subsection{Magnetic Moment}
Recall the vector expansion $\frac 1{\abs{\mathbf x-\mathbf x'}}=\frac 1{\abs{\mathbf x}}+\frac{\mathbf x\cdot\mathbf x'}{\abs{\mathbf x}^3}+\cdots$. A given component of the vector potential is 
$$A_i(\mathbf x)=\frac{\mu_0}{4\pi}\left[\frac 1{\abs{\mathbf x}}\int J_i(\mathbf x')\dd^3x'+\frac{\mathbf x}{\abs{\mathbf x}^3}\cdot\int J_i(\mathbf x')\mathbf x'\dd^3x'+\cdots\right]$$
Using the identity $f\mathbf J\cdot\nabla'g+g\mathbf J\cdot\nabla'f+fg\nabla'\cdot\mathbf J=0$ with $f=1,g=x'_i$, we have $\int J_i(\mathbf x')\dd^3x'=0$. With $f=x'_i,g=x'_j$, we have $\int(x'_iJ_j+x'_jJ_i)\dd^3x'=0.$ Hence, we can write the dipole term as 
$$\mathbf x\cdot\int\mathbf x'J_i\dd^3x'=-\frac 12\sum_jx_j\int(x'_iJ_j-x'_jJ_i)\dd^3x'=-\frac 12\left[\mathbf x\times\int(\mathbf x'\times\mathbf J)\dd^3x'\right]_i.$$
We can then define 
$$\mathcal M(\mathbf x)=\frac 12(\mathbf x\times\mathbf J(\mathbf x)),\quad\mathbf m=\frac 12\int\mathbf x'\times\mathbf J(\mathbf x')\dd^3x'\implies\mathbf A_\text{dipole}(\mathbf x)=\frac{\mu_0}{4\pi}\frac{\mathbf m\times\mathbf x}{\abs{\mathbf x}^3}\implies\mathbf B(\mathbf x)=\frac{\mu_0}{4\pi}\left[\frac{3\mathbf n(\mathbf n\cdot\mathbf m)-\mathbf m}{\abs{\mathbf x}^3}\right].$$
The magnitude of a planar current loop is evidently $\abs{\mathbf m}=I\cdot(\text{Area})$. Lastly, at the origin, we write 
$$\mathbf B(\mathbf x)=\frac{\mu_0}{4\pi}\left[\frac{3\mathbf n(\mathbf n\cdot\mathbf m)-\mathbf m}{\abs{\mathbf x}^3}+\frac{8\pi}3\mathbf m\delta(0)\right],$$
similar to the electric dipole counterpart (\textcolor{red}{Jackson 5.6}).

In general, the force term is, with $B_k(\mathbf x)=B_k(0)+\mathbf x\cdot\nabla B_k(0)+\cdots$, 
$$F_i=\sum_{jk}\epsilon_{ijk}\int J_j(\mathbf x')\mathbf x'\cdot\nabla B_k(0)\dd^3x'+\cdots=\sum_{jk}\epsilon_{ijk}(\mathbf m\times\nabla)_jB_k(\mathbf 0)\implies\mathbf F=(\mathbf m\times\nabla)\times\mathbf B=\nabla(\mathbf m\cdot\mathbf B).$$
Likewise, 
$$\mathbf N=\mathbf m\times\mathbf B(0),\quad U=-\mathbf m\cdot\mathbf B.$$
This has many effects in quantum mechanics; hyperfine splitting for example.

\subsection{Macroscopic Media}
Defining the magnetization $\mathbf M(\mathbf x)=\sum_iN_i\ev{\mathbf m_i}$, we have 
$$\mathbf A(\mathbf x)=\frac{\mu_0}{4\pi}\int\left[\frac{\mathbf J(\mathbf x')}{\abs{\mathbf x-\mathbf x'}}+\frac{\mathbf M(\mathbf x')\times(\mathbf x-\mathbf x')}{\abs{\mathbf x-\mathbf x'}^3}\right]\dd^3x'=\frac{\mu_0}{4\pi}\int\frac{[\mathbf J(\mathbf x')+\nabla'\times\mathbf M(\mathbf x')]}{\abs{\mathbf x-\mathbf x'}}\dd^3x'.$$
We define the magnetic field strength $\mathbf H$:
$$\mathbf H=\frac 1{\mu_0}\mathbf B-\mathbf M\implies\nabla\times\mathbf H=\mathbf J.$$

The boundary conditions are such that 
$$(\mathbf B_2-\mathbf B_1)\cdot\mathbf n=0,\quad\mathbf n\times(\mathbf H_2-\mathbf H_1)=\mathbf K.$$

\paragraph{Magnetization Problems}
With $\mathbf M$ given and $\mathbf J=0$, we have Poisson's equation;
$$\Phi_M(\mathbf x)=-\frac 1{4\pi}\nabla\cdot\int\frac{\mathbf M(\mathbf x')}{\abs{\mathbf x-\mathbf x'}}\dd^3x'=\frac{\mathbf m\cdot\mathbf x}{4\pi r^3}.$$
If there's a discontinuity in the magnetization at the surface, we write
$$\Phi_M(\mathbf x)=-\frac 1{4\pi}\int_V\frac{\nabla'\cdot\mathbf M(\mathbf x')}{\abs{\mathbf x-\mathbf x'}}\dd^3x'+\frac 1{4\pi}\oint_S\frac{\mathbf n'\cdot\mathbf M(\mathbf x')\dd a'}{\abs{\mathbf x-\mathbf x'}},$$
with $n\cdot\mathbf M$ acting as the magnetic surface-charge density.

For example, \textcolor{red}{Jackson 5.10, 5.11, 5.12}

\section{Electrodynamics}
While the statics solution was valid for $\nabla\cdot\mathbf J=0$, by making the generalization $\mathbf J\to\mathbf J+\pde{\mathbf D}t$, we get 
$$\nabla\cdot\mathbf D=\rho,\quad\nabla\cdot\mathbf B=0,\quad\nabla\times\mathbf H=\mathbf J+\pde{\mathbf D}t,\quad\nabla\times\mathbf E=-\pde{\mathbf B}t.$$

\subsection{Potentials}
The magnetic Gauss' and Faraday's law can be solved with scalar and vector potentials:
$$\mathbf B=\nabla\times\mathbf A,\quad\mathbf E=-\nabla\Phi-\pde{\mathbf A}t.$$
We can make a gauge transformation $\mathbf A\to\mathbf A+\nabla\Lambda$ and $\Phi\to\Phi-\pde{\Lambda}t$. If we choose the gauge such that $\nabla\cdot\mathbf A+\frac 1{c^2}\pde{\Phi}t=0$, we get 
$$\nabla^2\Phi-\frac 1{c^2}\pden 2\Phi t=-\rho/\epsilon_0,\quad\nabla^2\mathbf A-\frac 1{c^2}\pden 2{\mathbf A}t=-\mu_0\mathbf J.$$
Furthermore, $\Lambda$ that satisfies the Lorenz condition belongs to the Lorenz gauge. 

\subsection{Poynting's theorem}
The rate of work (no $B$!) can be written as an integral of $\mathbf J\cdot\mathbf E$. If the medium is linear in $\mu,\epsilon$, we define the energy density and obtain
$$u=\frac 12(\mathbf E\cdot\mathbf D+\mathbf B\cdot\mathbf H)\implies\pde ut+\nabla\cdot\mathbf S=-\mathbf J\cdot\mathbf E.$$
$\mathbf S$, the energy flow, is called the Poynting vector. Poynting's theorem expresses 
$$\ode Et=-\oint_S\mathbf n\cdot\mathbf S\dd a=\ode{}t(E_\text{mech}+E_\text{field}),\quad E_\text{field}=\frac{\epsilon_0}2\int_V(\mathbf E^2+c^2\mathbf B^2)\dd^3x,\quad E_\text{mech}=\int_V\mathbf J\cdot\mathbf E\dd^3x.$$
Similarly, 
$$\sum_\beta\int_V\pde{}{x_\beta}T_{\alpha\beta}\dd^3x=\oint_S\sum_\beta T_{\alpha\beta}n_\beta\dd a=\ode{}t(\mathbf P_\text{mech}+\mathbf P_\text{field})_\alpha,$$
where
$$\ode{\mathbf P}t=\int_V(\rho\mathbf E+\mathbf J\times\mathbf B)\dd^3x,\quad \mathbf P_\text{field}=\mu_0\epsilon_0\int_V\mathbf E\times\mathbf H\dd^3x,\quad T_{\alpha\beta}=\epsilon_0\left[E_\alpha E_\beta+c^2B_\alpha B_\beta-\frac 12(\mathbf E\cdot\mathbf E+c^2\mathbf B\cdot\mathbf B)\delta_{\alpha\beta}\right].$$
Note that the momentum density is $\mathbf g=\frac 1{c^2}\mathbf E\times\mathbf H=\frac 1{c^2}\mathbf S.$

\section{Lecture 24 (Nov. 19) - Radiation and Causality}

Maxwell's equations:
$$\partial_\mu F^{\mu\nu}=\frac{4\pi}cJ^\nu.$$
In Lorenz gauge, 
$$\partial_\mu A^\mu=0\implies\partial_\mu F^{\mu\nu}=\Box A^\mu-\partial^\nu(\Ccancel[red]{\partial_\mu A^\mu})\implies \Box A^\nu=\frac{4\pi}cJ^\nu.$$
This is the electromagnetic wave equation.

\paragraph{Green's Function}
Let us solve for the Green's function. The equation we'd like to solve is 
$$\Box_zD(z)=\delta^{(4)}(z).$$
Working in the momentum space,
$$D(z)=\frac 1{(2\pi)^4}\int\dd^4k\tilde D(k)e^{-ikz},\quad\delta^{(4)}=\frac 1{(2\pi)^4}\int\dd^4ke^{-ikz},\quad\Box=-k_\mu k^\mu,$$
we obtain 
$$\tilde D(k)=-\frac 1{k_\mu k^\mu}\implies D(z)=-\frac 1{(2\pi)^4}\int\dd^3k_se^{-i\mathbf k\cdot\mathbf z}\int_{\mathbb R}\dd k_0\frac{e^{-ikz_0}}{k^2_0-k^2_s}.$$
For the advanced propagator prescription ($\dd k_0$ integral runs above the real axis), we have, 
\begin{align*}
    \int_{\mathbb R}\dd k_0\frac{e^{-ikz_0}}{k_0^2-k_s^2}=\begin{cases}0&z_0<0\\\frac{2\pi}{k_s}\sin(k_sz_0)&z_0>0\end{cases}.
\end{align*}
Note that $z_0<0$ corresponds to closing the contour on the upper half plane, and $z_0>0$ corresponds to closing the contour on the lower half plane. Therefore, 
\begin{align*}
    D(z)&=\frac 1{(2\pi)^4}\int\dd^3k_s\Theta(z_0)\left[\frac{2\pi}{k_s}\sin(k_sz_0)\right]e^{-i\mathbf k_x\cdot\mathbf z}\\
    &=\frac{\Theta(z_0)}{(2\pi)^3}\int\dd k_sk_s^2\sin\theta\dd\theta\dd\phi e^{-ik_sz\cos\theta}\frac{\sin(k_sz_0)}{k_s}\\
    &=\frac{\Theta(z_0)}{2\pi^2z}\int^\infty_{-\infty}\dd k_s\sin(k_sz)\sin(k_sz_0)\\
    &=\frac{\Theta(z_0)}{4\pi z}\delta(z_0-z).
\end{align*}
Note that I've been using $z$ as $\abs{\mathbf z}$.

With $\delta(z_0^2-z^2)=\frac 1{2z}[\delta(z_0+z)+\delta(z_0-z)]$, we can finally write
$$D(z)=\frac{\Theta(z_0)\delta(z_\mu z^\mu)}{2\pi}.$$

\paragraph{Moving Charge, Causality}
Consider a moving charge. Its four-current can be written 
$$J^\nu=e\int\dd\tau\cdot cv^\nu(\tau)\delta^{(4)}(x-r(\tau)).$$
Using the Green's function, we obtain
\begin{align*}
    A^\nu(x)&=\frac{4\pi}c\int\dd^4x'J^\nu(x')D(x-x')\\
    &=\frac{4\pi}c\int\dd^4x'\Theta(x_0-x'_0)\delta((x-x')^2)\int\dd\tau\cdot cev^\alpha(\tau)\delta^{(4)}(x'-r(\tau))\\
    &=2\int\dd\tau\cdot cev^\nu(\tau)\cdot\Theta(x_0-r_0(\tau))\cdot\delta((x-r(\tau))^2)
\end{align*}
We note that 
\begin{itemize}
    \item The delta function ensures that only light-like separation contributes to $A$;
    \item The theta function ensures causality.
\end{itemize}

\paragraph{Lienard-Wiechert Potential}
The potential of a moving charge is given 
$$A^\mu=-e\frac{v^\nu}{v_\mu(x-r(\tau)^\mu}$$

\end{document}